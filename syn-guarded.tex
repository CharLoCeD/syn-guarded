\documentclass[a4paper,10pt]{amsart}

\usepackage{amsmath,amssymb,amsthm,amsfonts}
\usepackage[numbers]{natbib}
\usepackage[english]{babel}
\usepackage[X2,T1]{fontenc}
\usepackage{libertine}
\usepackage{libertinust1math}
\usepackage[utf8]{inputenc} %'utf8' instead of 'latin1'
% \usepackage{enumitem}
\usepackage{enumerate}
\usepackage{tikz-cd}
\usepackage{hyperref}
\usepackage{stmaryrd}
\usepackage{appendix}
\usepackage{mathtools}
\usepackage{cleveref}
\usepackage{quiver}

\tikzcdset{pullback/.style = {"\lrcorner"{anchor = center, pos = 0.125}, draw = none}}

\newtheorem{theorem}{Theorem}[section]
\newtheorem{fact}[theorem]{Fact}
\newtheorem{lemma}[theorem]{Lemma}
\newtheorem{conjecture}[theorem]{Conjecture}
\newtheorem{corollary}[theorem]{Corollary}
\newtheorem{claim}[theorem]{Claim}
\newtheorem{proposition}[theorem]{Proposition}
\newtheorem{observation}[theorem]{Observation}

\theoremstyle{definition}
\newtheorem{example}[theorem]{Example}
\newtheorem{definition}[theorem]{Definition}
\newtheorem{remark}[theorem]{Remark}
\newtheorem{question}[theorem]{Open Question}
\newtheorem{assumption}[theorem]{Assumption}
\newtheorem{qquestion}[theorem]{Question}
\newtheorem*{axiom}{Axiom}
\newtheorem{construction}{Construction}

\newcommand{\mc}[1]{\mathcal{#1}}
\newcommand{\mb}[1]{\mathbf{#1}}
\newcommand{\mbb}[1]{\mathbb{#1}}
\newcommand{\T}{\mbb T}
\newcommand{\I}{\mbb I}
\newcommand{\hy}{\uv{\Sigma}}
\newcommand{\gn}[1]{\ulcorner\! #1 \!\urcorner}
\newcommand{\mr}[1]{\mathrm{#1}}
\newcommand{\mf}[1]{\mathfrak{#1}}
\newcommand{\ms}[1]{\mathsf{#1}}
\newcommand{\Ind}{\mathbf{Ind}}
\newcommand{\Pro}{\mathbf{Pro}}
\newcommand{\Set}{\mb{Set}}
\newcommand{\Prop}{\mb{Prop}}
\newcommand{\sSet}{\mb{sSet}}
\newcommand{\sSeto}{\mb{sSet}_{\le 1}}
\newcommand{\End}{\operatorname{End}}
\newcommand{\Hom}{\operatorname{Hom}}
\newcommand{\Fin}{\mb{Fin}}
\newcommand{\Cnt}{\mb{Cnt}}
\newcommand{\Enm}{\mb{Enm}}
\newcommand{\Grp}{\mb{Grp}}
\newcommand{\sub}{\mr{Sub}}
\newcommand{\Lex}{\mb{Lex}}
\newcommand{\Pos}{\mb{Pos}}
\newcommand{\Kp}{\mb{Kp}}
\newcommand{\IKp}{\mb{IKp}}
\newcommand{\Ep}{\mb{Ep}}
\newcommand{\Kpp}{\Kp_{*}}
\newcommand{\alg}{\text{-}\mb{Alg}}
\newcommand{\Var}{\mb{Var}}
\newcommand{\Aff}{\mb{Aff}}
\newcommand{\Vect}{\mb{Vect}}
\newcommand{\CRing}{\mb{CRing}}
\newcommand{\DL}{\mb{DL}}
\newcommand{\BA}{\mb{BA}}
\newcommand{\HA}{\mb{HA}}
\newcommand{\JL}{\mb{JL}}
\newcommand{\ML}{\mb{ML}}
\newcommand{\ProMon}{\mb{ProMod}}
\newcommand{\CTopoi}{\mb{CTopoi}}
\newcommand{\PTc}{\mb{PT}_c}
\newcommand{\Topoi}{\mb{Topoi}}
\newcommand{\sh}{\mb{Sh}}
\newcommand{\psh}{\mb{Psh}}
\newcommand{\Cont}{\mb{Cont}}
\newcommand{\Cart}[1]{#1^\to_{\text{cart}}}
\newcommand{\Str}{\mb{Str}}
\newcommand{\op}{^{\mathrm{op}}}
\newcommand{\inv}{^{\mathrm{-1}}}
\newcommand{\pf}[1]{\widehat{#1}}
\newcommand{\qsi}[1]{\widetilde{#1}}
\newcommand{\cob}{\vartriangleleft}
\newcommand{\other}{\mathrm{otherwise}}
\newcommand{\geo}[1]{\left|#1\right|}
\newcommand{\ov}[1]{\overline{#1}}
\newcommand{\set}[1]{\{\,#1\,\}}
\newcommand{\eff}{\Leftrightarrow}
\newcommand{\conjt}{\;\&\;}
\newcommand{\pair}[1]{\left\langle#1\right\rangle}
\newcommand{\id}{\mathrm{id}}
\newcommand{\ev}{\mathrm{ev}}
\newcommand{\elem}{\int\!\!}
\newcommand{\nt}{\Rightarrow}
\newcommand{\scomp}[2]{\{\,#1\mid#2\,\}}
\newcommand{\yon}{\mathtt{y}}
\newcommand{\surj}{\twoheadrightarrow}
\newcommand{\inj}{\rightarrowtail}
\newcommand{\hook}{\hookrightarrow}
\newcommand{\cg}{\operatorname{\sim}}
\newcommand{\im}{\operatorname{img}}
\newcommand{\cgf}[2]{\leftindex_{#1}{\cg}_{#2}}
\newcommand{\coten}{\pitchfork}
\newcommand{\ppr}{\operatorname{\hat\times}}
\newcommand{\rl}{^{\perp}}
\newcommand{\llo}[1]{\leftindex_{}^{\perp} {#1}}
\newcommand{\dl}{^{\circ}}
\newcommand{\prth}[1]{\left(#1\right)}
\newcommand{\fn}{_{\mr{f.}}}
\newcommand{\lf}{_{\mr{l.f.}}}
\newcommand{\fp}{_{\mr{f.p.}}}
\newcommand{\fpp}{_{\mr{f.p.,+}}}
\newcommand{\cp}{_{\mr{c.p.}}}
\newcommand{\cpp}{_{\mr{c.p.,+}}}
\newcommand{\can}{_{\mr{can}}}
\newcommand{\pls}{^+}
\newcommand{\mns}{^-}
\newcommand{\dv}{\operatorname{\uparrow}}
\newcommand{\cv}{\operatorname{\downarrow}}
\newcommand{\et}{_{\text{\'et}}}
\newcommand{\rg}[1]{\mc O_{#1}}
\newcommand{\res}[1]{|_{#1}}
\newcommand{\N}{\mb N}
\newcommand{\Q}{\mbb Q}
\newcommand{\Z}{\mbb Z}
\newcommand{\Deltao}{\Delta_{\le 1}}
\newcommand{\Deltaw}{\Delta_{\omega}}
\newcommand{\sk}{\ms{sk}}
\newcommand{\csk}{\ms{csk}}
\newcommand{\sInt}{\mb{sInt}}
\newcommand{\jcan}{J_{\mr{can}}}
\newcommand{\wCPO}{\omega\mb{CPO}}
\newcommand{\shape}{\operatorname{\smallint}}
\newcommand{\dneg}{\neg\neg}
\newcommand{\prt}{_{\bot}}
\newcommand{\cprt}{_{\top}}
\newcommand{\fa}[2]{\forall #1\in #2.\ }
\newcommand{\ex}[2]{\exists #1\in #2.\ }
\newcommand{\exu}[2]{\exists_! #1\!\colon\!\!#2.\ }
\newcommand{\ld}[2]{\lambda #1\!\colon\!\!#2.\ }
\newcommand{\subopen}{\subseteq_{\mbb \I}}
\newcommand{\emp}{\emptyset}
\newcommand{\eq}{\leftrightarrow}
\newcommand{\ass}[1]{\llbracket#1\rrbracket} %\usepackage{stmaryrd}
\newcommand{\pss}[1]{||#1||} %\usepackage{stmaryrd}
\newcommand{\pt}{\ms{pt}}
\newcommand{\tp}{\ms{Type}}
\newcommand{\pp}{\ms{Prop}}
\newcommand{\st}{\ms{Set}}
\newcommand{\cnt}{\ms{Cnt}}
\newcommand{\gp}{\ms{Grpd}}
\newcommand{\pcat}{\ms{PCat}}
\newcommand{\cat}{\ms{Cat}}
\newcommand{\pcatt}{\ms{PCAT}}
\newcommand{\catt}{\ms{CAT}}
\newcommand{\PCat}{\mb{PCat}}
\newcommand{\Cat}{\mb{Cat}}
\newcommand{\sFrm}{\sigma\mb{Frm}}
\newcommand{\Frm}{\mb{Frm}}
\newcommand{\Loc}{\mb{Loc}}
\newcommand{\PCAT}{\mb{PCAT}}
\newcommand{\CAT}{\mb{CAT}}
\newcommand{\Catt}{\mf{Cat}}
\newcommand{\CATT}{\mf{CAT}}
\newcommand{\Topp}{\mb{Top}}
\newcommand{\Top}{\mf{Top}}
\newcommand{\wTop}{\omega\mb{Top}}
\newcommand{\Tp}{\ms{TYPE}}
\newcommand{\Pp}{\ms{PROP}}
\newcommand{\St}{\ms{SET}}
\newcommand{\Gp}{\ms{GRPD}}
\newcommand{\fst}{\ms{Fin}}
\newcommand{\Mod}{\mb{Mod}}
\newcommand{\Quot}{\mr{Quot}}
\newcommand{\Fil}{\mr{Fil}}
\newcommand{\SFil}{\mr{SFil}}
\newcommand{\quot}[1]{/_{\pair{#1}}}
\newcommand{\List}{\ms{List}}
\newcommand{\hp}{\text{-}}
\newcommand{\PG}{\ms{PG}}
\newcommand{\uv}[1]{\underline{#1}}
\newcommand{\mmod}[1]{#1\text{-}\mathbf{Mod}}
\newcommand{\func}{\mb{Func}}
\newcommand{\tm}[1]{#1\text{-}\mathrm{Term}}
\newcommand{\eqn}[1]{#1\text{-}\mathrm{Eqn}}
\newcommand{\horn}[1]{#1\text{-}\mathrm{Horn}}
\newcommand{\gr}[2]{[#1|#2]}
\newcommand{\VT}{\mbb V_\T}
\newcommand{\spec}{\operatorname{Spec}}
\newcommand{\El}{\mr{El}}
\newcommand{\lan}{\ms{lan}}
\newcommand{\ran}{\ms{ran}}
\newcommand{\upp}{_{\ms U}}
\newcommand{\dsg}[1]{\!\pair{#1}}
\newcommand{\subN}{\sub_{\N}}
\newcommand{\pw}{\mbb P_\omega}
\newcommand{\um}{\mbb w}
\newcommand{\prev}{\triangleleft}
\newcommand{\latt}{\triangleright}

\DeclareFontFamily{U}{dmjhira}{}
\DeclareFontShape{U}{dmjhira}{m}{n}{ <-> dmjhira }{}
\DeclareRobustCommand{\yon}{\text{\usefont{U}{dmjhira}{m}{n}\symbol{"48}}}
\DeclareRobustCommand{\noy}{\text{\reflectbox{\yon}}\!}


\makeatletter
\newcommand{\ct@}[2]{%
  \vtop{\m@th\ialign{##\cr
    \hfil$#1\operator@font lim$\hfil\cr
    \noalign{\nointerlineskip\kern1.5\ex@}#2\cr
    \noalign{\nointerlineskip\kern-\ex@}\cr}}%
}
\newcommand{\ct}{%
  \mathop{\mathpalette\ct@{\rightarrowfill@\textstyle}}\nmlimits@
}
\makeatother
\makeatletter
\newcommand{\lt@}[2]{%
  \vtop{\m@th\ialign{##\cr
    \hfil$#1\operator@font lim$\hfil\cr
    \noalign{\nointerlineskip\kern1.5\ex@}#2\cr
    \noalign{\nointerlineskip\kern-\ex@}\cr}}%
}
\newcommand{\lt}{%
  \mathop{\mathpalette\lt@{\leftarrowfill@\textstyle}}\nmlimits@
}
\makeatother

\title{Synthetic guarded domain theory and classifying topoi}
\author{Lingyuan Ye}
\address{
Lingyuan \textsc{Ye} \newline
Department of Computer Science and Technology\newline
University of Cambridge\newline
Cambridge, UK\newline
\href{mailto:ye.lingyuan.ac@gmail.com}{\sf ye.lingyuan.ac@gmail.com}
}

\begin{document}
%

%
%\titlerunning{Abbreviated paper title}
% If the paper title is too long for the running head, you can set
% an abbreviated paper title here
%

%
% \Endhorrunning{L. Ye}
% First names are abbreviated in the running head.
% If there are more than two authors, 'et al.' is used.
%
% \institute{New College\\
% \begin{abstract}

% \end{abstract}
%
\maketitle              % typeset the header of the contribution
%

\section{Introduction}

This is an internal analysis on synthetic guarded domain theory~\cite{birkedal2012first}.

\section{The theory of sequential propositions}

Let $\ov\omega$ be the poset of extended natural numbers $\N \cup \set{\infty}$. As a poset it is a meet-semi-lattice, thus corresponds to an essentially algebraic theory, which we call $\mbb P$. Since it is a lattice, it is also localic, and can be presented as a theory with infinitely many propositional letters $p_0,p_1,\cdots$, with infinitely many sequents as axioms
\[ p_0 \vdash p_1 \vdash p_2 \vdash \cdots \]
We also consider a geometric quotient $\pw$ of $\mbb P$, which is obtained by adding an additional geometric sequent
\[ \top \vdash \bigvee_{n:\N}p_n. \]

The classifying topos $\Set[\mbb P]$ is given by the presheaf category
\[ \Set[\mbb P] \simeq \psh(\ov\omega). \]
Inside, the universal model $\um$ is given by the sequence of representables,
\[ \um := \yon_0 \hook \yon_1 \hook \cdots \]
The classifying topos for $\pw$ is again of presheaf type, given by
\[ \Set[\pw] \simeq \psh(\omega). \]
As a subtopos $\psh(\omega) \hook \psh(\ov\omega)$, since $\bigvee_{n:\N}n = \infty$ is a universal effective colimit in $\ov\omega$, all representables are still sheaves in $\psh(\omega)$. Thus, the universal model $\um$ is again the generic model in $\psh(\omega)$. Below we will mainly work with the sheaf subtopos $\psh(\omega)$, and we will denote it as $\mc S$.

\begin{proposition}\label{endoposet}
  Let $\ov\omega\prt$ be the poset $\ov\omega$ with a bottom element added, and $\mb{Inf}(\omega,\ov\omega\prt)$ denote the poset of unbounded monotone functions from $\omega$ to $\ov\omega\prt$. Then we have
  \[ \Topoi(\mc S,\mc S) \simeq \mmod{\pw}(\mc S) \simeq \mb{Inf}(\omega,\omega\prt). \]
\end{proposition}
\begin{proof}
  The first equivalence is due to the universal property of the classifying topos. We note that a model of $\pw$ in $\mc S$ is a sequence of increasing propositions whose union is $1$. Furthermore, we know that in $\mc S$ we have $\sub(1) \cong \ov\omega\prt$, since a proposition can either be $\yon_k$ for some $k$, or $0,1$. This explains the second equivalence.
\end{proof}

We can describe the correspondence more concretely: For any $\mf F \in \mmod{\pw}(\mc S)$, it is a sequence of propositions in $\mc S$
\[ \mf F_0 \hook \mf F_1 \hook \cdots \]
such that $\bigvee_{n:\N}\mf F_n = 1$. It induces a geometric morphism
\[\begin{tikzcd}
  {\psh(\omega)} & {\psh(\omega)}
  \arrow[""{name=0, anchor=center, inner sep=0}, "\prev", curve={height=-12pt}, from=1-1, to=1-2]
  \arrow[""{name=1, anchor=center, inner sep=0}, "\latt", curve={height=-12pt}, from=1-2, to=1-1]
  \arrow["\dashv"{anchor=center, rotate=-90}, draw=none, from=0, to=1]
\end{tikzcd}\]
The right adjoint is easy to define, which is given by
\[ \latt X(k) \cong \mc S(\mf F_k,X) \cong X(f(k)), \]
where here we view $f : \omega \to \ov\omega\prt$ as $\mf F$ under the equivalence in \Cref{endoposet}, and we define the value of $X$ on $\bot,\infty$ to be
\[ X(\bot) := 1, \quad X(\infty) := \lt_{n:\N}X(n). \]

The left adjoint $\prev$ is a left Kan extension,
\[ \prev X \cong \ct_{x\in X(n)}\mf F_n. \]
However, we also have a more concrete description of $\prev$: For any $k\in\N$, we define
\[ k^p := \min\scomp{n:\N}{k \le f(n)}. \]
This is well-defined exactly because $\bigvee_{n:\N}\mf F_n = 1$. 

\begin{lemma}\label{prevpointwise}
  For any presheaf $X \in\mc S$,
  \[ \prev X(k) \cong X(k^p). \]
\end{lemma}
\begin{proof}
  By construction, since colimits in $\mc S$ are computed pointwise,
  \[ \prev X(k) \cong \ct_{x\in X(n)}\mc S(\yon_k,\mf F_n). \]
  Notice that $\mc S(\yon_k,\mf F_n) \cong 1$ when $k \le f(n)$, and is empty otherwise. This makes $X(k^p)$ terminal in the above colimit.
\end{proof}

\begin{example}
  The universal model $\um$ induces the identity on $\mc S$.
\end{example}

\begin{example}
  Consider the model $\um[-1]$ defined by 
  \[ \um[-1] := 0 \hook \yon_0 \hook \yon_1 \hook \cdots \]
  The adjoint pair $\prev \dashv \latt$ induced by $\um[-1]$ is the usual modalities used in synthetic guarded domain theory~\cite{birkedal2012first}.
\end{example}

\section{Guarded recursion externally}

\begin{definition}
  We say a model $\mf F$ is \emph{subgeneric} if $\mf F \le \um$.
\end{definition}

Under the equivalence in \Cref{endoposet}, a subgeneric model $\mf F$ induces the following natural transformations,
\[ \sigma : \prev \nt \id, \quad \eta : \id \nt \latt. \]
Using $\eta$, we can in fact turn $\latt$ into an $\mc S$-indexed functor:

\begin{proposition}
  Let $\latt_X : \mc S/X \to \mc S/X$ sends $f : Y \to X$ to the following pullback,
  \[ 
  \begin{tikzcd}
    \latt_XY \ar[d, "\latt_Xf"'] \ar[dr, pullback] \ar[r] & \latt Y \ar[d, "\latt f"] \\ 
    X \ar[r, "\eta"'] & \latt X 
  \end{tikzcd}
  \]
  This gives a well-defined $\mc S$-indexed functor.
\end{proposition}
\begin{proof}
  See~\cite{birkedal2012first}.
\end{proof}

\begin{definition}
  We say a model $\mf F$ is \emph{inductive}, if $f(n) < n$ for all $n \in \omega$.
\end{definition}

\begin{theorem}
  For any inductive model $\mf F$, L\"ob's induction holds in $\mc S$, 
  \[ \fa\varphi\Omega (\latt\varphi \to \varphi) \to \varphi. \]
\end{theorem}
\begin{proof}
  Using the Kripke-Joyal semantics, suppose we have $\varphi \in \Omega(n)$ such that
  \[ n \models \latt\varphi \to \varphi. \]
  Now we have 
  \[ n \models \varphi \eff n \models \latt\varphi \eff f(n) \models \varphi \eff f(f(n)) \models \varphi \cdots \eff \bot \models \varphi. \]
  which then implies that $n \models \varphi$ since $\bot \models \varphi$ always holds.
\end{proof}

\section{Guarded recursion internally}

Equivalently, the geometric theory $\pw$ is the theory of \emph{filters} on $\omega$, and for any filter $\mf F$, the proposition $\mf F_n$ can be identified as $n \in \mf F$. In this section we will solely use the latter notation. 

\begin{definition}
  We say a filter $\mf F$ is \emph{sup-generic}, if $\um \subseteq \mf F$. The \emph{spectrum} of a sup-generic filter $\mf F$ is simply 
  \[ \spec \mf F := \mf F \subseteq \um \eq \mf F = \um. \]
\end{definition}

\begin{proposition}\label{qc}
  For any sup-generic filter $\mf F$, we have
  \[ \mf F = \um^{\spec\mf F}. \]
  In other words for any $n : \omega$,
  \[ n \in \mf F \eff \mf F = \um \to n \in \um. \]
\end{proposition}
\begin{proof}
  This is exactly quasi-coherence for the theory $\mbb P$.
\end{proof}

\begin{theorem}\label{filandprop}
  There is an adjunction between $\Omega$ and $\SFil(\omega)$ of sup-generic filters on $\omega$,
  \[\begin{tikzcd}
    {\SFil(\omega)\op} & \Omega
    \arrow[""{name=0, anchor=center, inner sep=0}, "\spec"', curve={height=18pt}, from=1-1, to=1-2]
    \arrow[""{name=1, anchor=center, inner sep=0}, "{\um^-}"', curve={height=18pt}, from=1-2, to=1-1]
    \arrow["\dashv"{anchor=center, rotate=-90}, draw=none, from=1, to=0]
  \end{tikzcd}\]
  which identifies $\SFil(\omega)\op$ as a reflective sub-poset of $\Omega$.
\end{theorem}
\begin{proof}
  For any proposition $\varphi : \Omega$ and sup-generic filter $\mf F$, if $\varphi \to \spec\mf F$, then 
  \begin{align*}
    n \in \mf F 
    &\eff \spec\mf F \to n \in \um \\
    &\nt \varphi \to n \in \um \\
    &\eff n \in \um^\varphi
  \end{align*}
  which implies $\mf F \subseteq \um^\varphi$. On the other hand, suppose $\mf F \subseteq \um^\varphi$. If $\varphi$ holds, then $\um^\varphi = \um$, which implies $\mf F = \um$ since $\mf F$ is sup-generic. Hence, $\spec\mf F$ also holds.
\end{proof}

\begin{definition}
  We say a proposition is \emph{affine} if it belongs to the image of $\spec$. By \Cref{filandprop}, affine propositions are 
\end{definition}

\begin{example}
  
\end{example}

\begin{proposition}
  Affine propositions are closed under arbitrary meets.
\end{proposition}
\begin{proof}
  
\end{proof}



\begin{remark}
  The above explains perfectly what it means for $\um$ to be the \emph{generic filter}.
\end{remark}

\begin{proposition}
  For any $n : \omega$, $\neg\neg n \in \um$.
\end{proposition}
\begin{proof}
  Suppose $n \not\in \um$. Then consider the filter $\mf P_n$ defined by
  \[ \mf P_n := \scomp{k : \omega}{n \le k}. \]
  For any $k \in \um$ we know that $n \le k$, since if $k < n$ and $k \in \um$ then $n \in \um$, contradictory. Hence, $n \le k$, since this proposition is decidable. It follows that $\mf P_n$ is sup-generic, which by \Cref{qc} implies for any $k : \omega$
  \[ k \in \mf P_n \eff n \le k \eff \mf P_n = \um \to k \in \um. \]
\end{proof}


Furthermore, the topos $\mc S$ can be viewed as an internal way to doing \emph{forcing} on $\omega$: 

\begin{definition}
  For any $n : \omega$ and $\varphi : \Omega$, we define the forcing relation
  \[ n \Vdash \varphi := \um_n \to \varphi. \]
\end{definition}

\begin{lemma}
  For any $n : \omega$, the forcing relation $n \Vdash (-) : \Omega \to \Omega$ has a right adjoint.
\end{lemma}
\begin{proof}
  Given a family $\varphi : I \to \Omega$ of propositions, consider the sup-generic filter $\Phi$,
  \[ n \in \Phi := n \in \um \vee \bigvee_{i:I}\varphi_i. \]
  This way, we have
  \[ \Phi = \um \eff \prth{\bigvee_{i:I}\varphi_i} \to \um_n = \bigwedge_{i:I} \varphi_i \to \um_n. \]
  By \Cref{qc}, we have that
  \[ n \in \Phi \eff \um_n \vee \bigvee_{i:I}\varphi_i \eff \prth{\bigwedge_{i:I}\varphi_i \to \um_n} \to \um_n. \]
  \[ n \Vdash \bigvee_{i:I}\varphi_i \eff \um_n \to \bigvee_{i:I}\varphi_i. \]
\end{proof}


This inspires the following construction:

\begin{construction}
  Any $\pw$-model $\mf F$ induces two modalities on $\Omega$, where for $\varphi : \Omega$
  \[ \prev\varphi := \bigvee_{\um_n \to \varphi} \mf F_n, \quad \latt\varphi := . \]
\end{construction}

\begin{lemma}
  For all $n : \omega$, $\prev\um_n = \mf F_n$.
\end{lemma}
\begin{proof}
  Notice that $\um_n \to \um_n$, thus $\mf F_n \to \bigvee_{\um_m\to\um_m}\mf F_m = \prev\um_n$. On the other hand, for any $m$ that $\um_m \to \um_n$, we have $m \le n$. Thus, $\mf F_m \to \mf F_n$, and $\prev\um_n = \bigvee_{\um_m \to \um_n}\mf F_m \to \mf F_n$.
\end{proof}

\begin{proposition}
  The two modalities are adjoint to each other $\prev \dashv \latt$.
\end{proposition}
\begin{proof}
  For any $\varphi,\psi : \Omega$, we have 
  \begin{align*}
    \prev\varphi \le \psi 
    &\eff \fa k\N (\um_n \to \varphi) \to (\mf F_n \to \psi). \\
    &\eff \fa k\N (\um_n \to \varphi) \wedge \mf F_n \to \psi
  \end{align*}
  On the other hand,
  \[ \varphi \le \latt\psi \eff \fa k\N p_k \subseteq \psi \to k \in \varphi. \]
\end{proof}

\begin{lemma}
  For any 
  \[ p \eq \um^{\spec p}. \]
\end{lemma}
\begin{proof}
  This holds by quasi-coherence.
\end{proof}

\begin{theorem}
  For any $\varphi : \Omega$, we have 
  \[ \varphi \vee \neg\varphi \vee \ex n\N \varphi = \um_n. \]
\end{theorem}
\begin{proof}
  
\end{proof}



\begin{definition}
  By an $\um$-algebra we mean an increasing proposition $p : \N \to \Omega$ such that $\um \le p$. The \emph{spectrum} of $p$ is defined to be
  \[ \spec p := \fa n\N p_n \le \um_n \eq p = \um. \]
\end{definition}




\begin{example}
  For $\um_n$, we have 
  \[ \prev\um_n = \bigvee_{\um_n \le p_k} \um_k \]
\end{example}

\bibliographystyle{apalike} 
\bibliography{mybib}

\end{document}